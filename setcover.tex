\chapter{Set Cover problém a výber génov}
\section{Definícia Set Cover problému}
Pre dané Univerzum $U$, ktoré obsahuje $n$ prvkov, a sadu podmnožín univerza \(S=\{ S_i : S_i \subseteq U \}\)
, ktorá pokrýva celé univerzum \(\{\cup S = U\}\),vybrať čo najmenšiu pod-sadu \(C \subseteq S\) takú že jej zjednotenie pokryje celé Univerzum 
\(\cup_{S \epsilon C} S = U \) .\\ Set Cover problém patrí medzi NP tažké problémy.
\section{Výber génov}
Pri analýze \emph{Evolučnej histórie} nás zaujímajú udalosti ktoré sa v nej odohrali. Veľké množstvo génov v evolučnej histórii môže viesť k neprehľadnosti a nemožnosti udalosť na obrázku identifikovať.
Z obrázku teda potrebujeme odstrániť prebytočné gény tak, aby sme nestratili informácie o udalostiach ktoré sa v histórii odohrali.
\paragraph{Pokrytie blokov}
Gény ktoré zobrazíme budeme vyberať na základe toho, koľko blokov zostane pokrytých.
Blok považujeme za pokrytý pokiaľ sa v zobrazení nachádza aspoň jeden gén z daného bloku.
Pokrytie bloku nám zároveň zachová približnú vizuálnu informáciu o tom aká udalosť sa tu odohráva.
\subsection{Výber génov pomocou Set Coveru}
Výber takých génov ktoré pokryjú všetky bloky vieme riešiť ako set cover problém.
Univerzum predstavuje všetky bloky ktoré sa nachádzajú v našom fylogenetickom strome.
Každý gén predstavuje jednu podmnožinu, v ktorej sa nachádzajú tie bloky, cez ktoré gén prechádza.
Riešením je taká množina génov, ktoré dokopy pokrývajú všetky bloky nachádzajúce sa v evolučnej histórii.
\section{Riešenie Set Cover problému}
\subsection{Greedy algoritmus}
Greedy algoritmus nám umožnuje v polynomiálnom čase nájsť približné riešenie set cover problému.
% Peter Slavik A tight analysis of the greedy algorithm for set cover
\paragraph{Algoritmus} Zoradíme si všetky podmnožiny na základe toho koľko prvkov obsahujú.
Do riešenia vyberieme najvačšiu podmnožinu, prvky ktoré sa v nej nachádzajú odstránime z Univerza aj zo zvyšných podmnožín, tie opäť zoradíme a postup opakujeme až pokým nieje Univerzum prázdne.
%ok pseudo code by bol asi lepsi
\subsection{ILP}
Náš Set Cover problém zapíšeme/pretransformujeme ako \emph{Integer Linear Programming} problém
