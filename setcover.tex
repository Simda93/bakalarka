\chapter{Problém množinového pokrytia a výber génov}
V predchádzajúcej kapitole sme si predstavili základné prvky nášho programu ktorý dostane na vstupe súbor, 
popisujúci evolučnú históriu a na výstupe nám vykreslí fylogenetický strom reprezentujúci danú históriu.
Problém nastane, pokiaľ v histórii nachádza príliš veľa génov. Výsledný vygenerovaný obrázok sa stáva neprehľadným, 
a získanie informácie z neho obtiažne. 
Potrebujeme teda vybrať iba niektoré gény na zobrazenie tak, aby na obrázku zostali zachované podstatné informácie.
V tejto kapitole si predstavíme spôsob, akým budeme vyberať ktoré gény zobrazíme,
využitie \emph{Problému množinového pokrytia} pri hľadaní daných génov a dva algoritmy ktoré riešia daný problém.
\section{Výber génov}
Najpodstatnejšou informáciou pri analýze fylogenetického stromu je pre nás to, aké udalosti sa v ňom odohrali. 
Budeme sa teda snažiť nájsť podmnožinu všetkých génov tak, aby všetky udalosti ostali na obrázku zachované.
Zvyšné gény následne z obrázku odstránime, čo môže viesť k strate informácií ktoré považujeme za menej podstatné, 
ako napríklad to, koľko a ktoré geńy sa nachádzajú v danej histórii, ako aj koľko a ktoré gény sú ovplyvné danou udalosťou.
\subsection{Blok}
\emph{Blok} predstavuje postupnosť génov ktoré sa pred aj po kroku e.h. nachádzali vedľa seba v rovnakom poradí a jednotlivé gény nemenili svoju orientáciu. Jedná sa teda o súvislý
úsek DNA ktorý počas kroku e.h. nebol prerušený.
Ak sa pri delécii alebo inzercii odobralo alebo pridalo viacero génov, a nenachádza sa medzi nimi žiaden iný gén, tvoria jeden blok.
Pri duplikácii gény tvoria blok ak sa nachádzali pri sebe pred duplikáciou a rovnako aj po nej vo všetkých zduplikovaných inštanciách.
Pri Inverzii sa gény nachádzajú v bloku pokiaľ sa všetkým zmení orientácia, t.j. zrotuje celý blok.
Napr blok génov (4,5,-6) bude po inverzii vyzerať ako (6,-5,-4).
\subsubsection{Pokrytie blokov}
Blok považujeme za pokrytý pokiaľ sa na obrázku vyskytuje aspoň jeden gén patriaci do daného bloku.
Pokrytie všetkých blokov jedného kroku e.h nám zaručí zobrazenie všetkých udalostí, aj keď nie v úplnom rozsahu,
ktoré sa v danom kroku vyskytli.
Musíme preto nájsť také gény, ktoré pokryjú všetky bloky v kompletnej evolučnej histórii,
a tým si zaistiť zobrazenie všetkých udalostí vo výslednom fylogenetickom strome. 
Ako cieľ si zvolíme aby bola daná množina génov čo najmenšia.
\section{Problém Množinového Pokrytia}
\paragraph{Definícia}
Máme dané univerzum$U$, ktoré obsahuje $n$ prvkov, a systém jeho podmnožín \(S=\{ P_i : P_i \subseteq U \}\), 
ktorý pokrýva celé univerzum \(\cup_{Pi \epsilon S} Pi = U\),
vybrať čo najmenšiu množinu podmnožín \(C \subseteq S\)
takú ktorá tiež pokryje celé Univerzum \(\cup_{Pi \epsilon C} Pi = U \) .\\
Problém Množinového Pokrytia (anglicky Set Cover Problem), ďalej len \emph{PMP}. patrí medzi NP-úplne problémy.%citation
\paragraph{Príklad}
Pre Univerzum$U=\{1,2,3,4,5,6\}$ \\a systém jeho podmnožín
$S=\{\{1,2,3\},\{2,3\},\{3,4\},\{3,4,6\},\{5\}\}$ \\
je riešením množina podmnožín $C=\{\{1,2,3\},\{3,4,6\},\{5\}\}$ 
\subsection{Výber génov pomocou Problému Množinového Pokrytia}
Výber takých génov ktoré pokryjú všetky bloky v celej evolučnej histórii vieme formulovať
ako Problém Množinového Pokrytia 
Univerzum predstavuje všetky bloky ktoré sa nachádzajú v našom fylogenetickom strome.
Každý gén predstavuje jednu podmnožinu, v ktorej sa nachádzajú tie bloky, cez ktoré gén prechádza.
Riešením je taká množina génov, ktorých zjednotenie pokrýva všetky prvky Univerza, v našom prípade všetky bloky nachádzajúce sa v evolučnej histórii.
\section{Riešenie Problému Množinového Pokrytia}
Keďže \emph{PMP} patrí medzi NP-ťažké problémy, znamená to že zatiaľ neexistuje,
a možno nikdy ani nebude existovať algoritmus ktorý by dokázal nájsť riešenie v polynomiálnom čase.
Potrebujeme sa teda rozhodnúť, či je pre nás výhodnejšie hľadať najlepšie riešenie \emph{PMP} 
čo môže byť časovo náročné,
alebo sa uspokojíme s približním riešením, ktoré sme schopný nájsť aproximačným algorimtom v polynomiálnom čase,
a ktoré môže taktiež predstavovať dostatočné odstránenie prebytočných génov z obrázku.
Predstavíme si jeden spôsob ktorým budeme hladať úplné riešenie, jeden spôsob na nájdenie približného riešenia a v
nasledujúcej kapitole porovnáme výsledky ktoré produkujú. 
\subsection{Greedy algoritmus}
Greedy algoritmus patrí medzi najlepšie polynomiálne aproximačné algoritmy pre riešenie \emph{PMP}.\cite{Slavik}
Greedy algoritmus v každom kroku pridá do riešenia takú podmnožinu,
ktorá obsahuje najviac zatiaľ nepokrytých prvkov univerza. Riešenie teda hľadáme nasledovným spôsobom:\\
% Peter Slavik A tight analysis of the greedy algorithm for set cover %citation
Všetky podmnožiny zoradíme na základe toho, koľko prvkov obsahujú.
Do riešenia vyberieme najvačšiu podmnožinu a prvky, ktoré sa v nej nachádzajú odstránime z
univerza aj zo zvyšných podmnožín. Zvyšné podmnožiny opäť zoradíme podľa veľkosti,
a postup opakujeme až pokým nie je Univerzum prázdne.\\
Tesná analýza podľa Slavíka ukazuje, že aproximačný koeficient takéhoto riešenia je \(\ln m - \ln \ln m\ +\Theta(1) \) \cite{Slavik} 
kde \(m = |U|\).
\subsection{ILP}
Náš Set Cover problém zapíšeme/pretransformujeme ako \emph{Integer Linear Programming} problém
