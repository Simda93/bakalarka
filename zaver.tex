\chapter*{Záver}
\addcontentsline{toc}{chapter}{Záver}

Cieľom tejto práce bolo implementovať systém na vizualizáciu evolučných histórií, s dôrazom na zmeny, ktoré sa v nich odohrali.
Výsledkom je program EHDraw ktorý umožňuje jednoduchú manipuláciu za pomoci grafickej aplikácie.
Určený je najmä pre vedeckých pracovníkov, ktorý sa zaoberajú problémami z oblasti biológie a bioinformatiky,
konkrétne rekonštrukciou DNA sekvence alebo celého fylogenetického stromu. 
Ponúka im jednoduchú vizuálnu kontrolu ich práce, ako aj možnosť prezentovať svoje výsledky formou, ktorá dokáže osloviť širšie publikum.
Za týmto účelom umožňuje náš program zmenu veľkého počtu nastavení, používateľ získava kontrolu nad vzhľadom vyprodukovanej vizualizácie.
S dôrazom na zobrazenie všetkých zmien, riešime aj problém výberu génov. Jeho cieľom je nájsť takú sadu génov,
ktoré môžme zo zobrazenia odstráni bez toho, aby sme stratili informáciu o udalostiach v evolučnej histórií.
Odstránenie niektorých génov má uľahčiť nájdenie podstatnej informácie na obrázku.
Takúto sadu génov nájdeme pomocou Problému množinového pokrytia, 
pre jeho vyriešenie môžme využiť buď greedy algoritmus alebo Celočíselné lineárne programovanie.
Pri Celočíselnom lineárnom programovaní máme záruku nájdenia optima, zatiaľčo greedy hľadá aproximáciu.
Praktickými testami sme overovali či greedy algoritmus predstavuje časovú úsporu, a o koľko horšie riešenie od optima vyhľadá.
Prvotné predpoklady že greedy algoritmus nájde riešenie v kratšom čase sa nepotvrdili, v niektorých prípadoch bol greedy rýchlejší,
ale pri optimalizácii histórií, ktoré sú dosť malé na to, aby sme ich zobrazovali, nebudú takéto rozdiely v čase podstatné.
Najväčšie množstvo času,pri najväčších testovaných históriách oba algoritmy strávili v programe EHDraw, ktorému vygenerovanie lineárneho programu 
zo vstupu trvalo niekoľko krát dlhšie než jeho vyriešenie CPLEXom.
Na druhú stranu, riešenia ktoré nachádzal greedy boli iba málo rozdielne od optím. V praxi to znamená jeden alebo dva gény naviac v zobrazení.

Táto práca nerieši všetky problémy súvisiace so zobrazovaním evolučných histórií, 