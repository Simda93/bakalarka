\documentclass[12pt,oneside,a4paper]{book}
\usepackage[a4paper,top=2.5cm,bottom=2.5cm,left=3.5cm,right=2cm]{geometry}
\usepackage[utf8]{inputenc}
\usepackage[T1]{fontenc}
\usepackage{graphicx}
\usepackage{url}
\usepackage[slovak]{babel} % vypnite pre prace v anglictine
\linespread{1.25} % hodnota 1.25 by mala zodpovedat 1.5 riadkovaniu
\overfullrule=2cm

% -------------------
% --- Definicia zakladnych pommov
% -------------------
\def\*{{\bf FIXME: }}
\def\mfyear{2016}
\def\mftitle{Vizualizácia evolučných histórií}
\def\mfthesistype{Bakalárska práca}
\def\mfauthor{Dávid Simeunovič}
\def\mfadvisor{doc. Mgr. Bronislava Brejová, PhD.}
\def\mfplacedate{Bratislava, \mfyear}
\def\odbor{2508 Informatka} %aj cislo odboru je povinne a je to podla katedry/odboru, na ktorom je autor
\def\program{ Informatika }
\def\stredisko{ Katedra Informatiky }

\def\mfrok{2016}
\def\mfnazov{Vizualizácia evolučných histórií}
\def\mftyp{Bakalárska práca}
\def\mfautor{Dávid Simeunovič}
\def\mfskolitel{doc. Mgr. Bronislava Brejová, PhD.}

%ak mate konzultanta, odkomentujte aj jeho meno na titulnom liste
\def\mfkonzultant{tit. Meno Priezvisko, tit. }  

\def\mfmiesto{Bratislava, \mfrok}

%aj cislo odboru je povinne a je podla studijneho odboru autora prace
\def\mfodbor{2508 Informatika} 
\def\program{ Informatika }
\def\mfpracovisko{ Katedra informatiky }

\begin{document}     

% -------------------
% --- Obalka ------
% -------------------
\thispagestyle{empty}

\begin{center}
\sc\large
Univerzita Komenského v Bratislave\\
Fakulta matematiky, fyziky a informatiky

\vfill

{\LARGE\mfnazov}\\
\mftyp
\end{center}

\vfill

{\sc\large 
\noindent \mfrok\\
\mfautor
}

\eject % EOP i
% --- koniec obalky ----

% -------------------
% --- Titulný list
% -------------------

\thispagestyle{empty}
\noindent

\begin{center}
\sc  
\large
Univerzita Komenského v Bratislave\\
Fakulta matematiky, fyziky a informatiky

\vfill

{\LARGE\mfnazov}\\
\mftyp
\end{center}

\vfill

\noindent
\begin{tabular}{ll}
Študijný program: & \program \\
Študijný odbor: & \mfodbor \\
Školiace pracovisko: & \mfpracovisko \\
Školiteľ: & \mfskolitel \\
% Konzultant: & \mfkonzultant \\
\end{tabular}

\vfill


\noindent \mfmiesto\\
\mfautor

\eject % EOP i


% --- Koniec titulnej strany


% -------------------
% --- Naskenovane Zadanie
% -------------------
% v tlačenej verzii s podpismi zainteresovaných osôb.
% v elektronickej verzii sa zverejňuje zadanie bez podpisov zainteresovaných osôb.

\newpage 
\thispagestyle{empty}
\hspace{-2cm}\includegraphics[width=1.1\textwidth]{images/zadanie}

% --- Koniec zadania

\frontmatter

% -------------------
%   Poďakovanie - nepovinné
% -------------------
\setcounter{page}{3}
\newpage 
~

\vfill
{\bf Poďakovanie:}


% --- Koniec poďakovania

% -------------------
%   Abstrankt - Slovensky
% -------------------
\newpage 
\section*{Abstrakt}

Počas evolúcie dochádza v DNA k lokálnym mutáciám, 
ktoré menia jeden alebo niekoľko susedných nukleotidov, 
ale aj k väčším zmenám, ktoré menia poradie alebo počet výskytov dlhších oblastí.
Cieľom práce je implementovať systém na vizualizáciu evolučnej histórie jednej 
alebo viacerých DNA sekvencií s dôrazom na tieto väčšie zmeny. Samotná história 
je daná na vstupe a cieľom je zobraziť ju tak, aby sa dali prehľadne sledovať jednotlivé
mutácie a tiež vzťahy rôznych častí sekvencie.


\paragraph*{Kľúčové slová:} vizualizácia,evolučná história,poradie génov
% --- Koniec Abstrakt - Slovensky


% -------------------
% --- Abstrakt - Anglicky 
% -------------------
\newpage 
\section*{Abstract}

Abstract in the English language (translation of the abstract in the
Slovak language).


\paragraph*{Keywords:} 

% --- Koniec Abstrakt - Anglicky

% -------------------
% --- Predhovor - v informatike sa zvacsa nepouziva
% -------------------
%\newpage 
%\thispagestyle{empty}
%
%\huge{Predhovor}
%\normalsize
%\newline
%Predhovor je všeobecná informácia o práci, obsahuje hlavnú charakteristiku práce 
%a okolnosti jej vzniku. Autor zdôvodní výber témy, stručne informuje o cieľoch 
%a význame práce, spomenie domáci a zahraničný kontext, komu je práca určená, 
%použité metódy, stav poznania; autor stručne charakterizuje svoj prístup a svoje 
%hľadisko. 
%
% --- Koniec Predhovor


% -------------------
% --- Obsah
% -------------------

\newpage 

\tableofcontents

% ---  Koniec Obsahu

% -------------------
% --- Zoznamy tabuliek, obrázkov
% -------------------

\newpage 

\listoffigures

% ---  Koniec Zoznamov

\mainmatter


\input uvod.tex 

\input biolog.tex

\input program.tex

\input setcover.tex

\input zaver.tex

% -------------------
% --- Bibliografia
% -------------------


\newpage	

\backmatter

\thispagestyle{empty}
\nocite{*}
\clearpage

\bibliographystyle{plain}
\bibliography{literatura} 

%Prípadne môžete napísať literatúru priamo tu
%\begin{thebibliography}{5}
 
%\bibitem{br1} MOLINA H. G. - ULLMAN J. D. - WIDOM J., 2002, Database Systems, Upper Saddle River : Prentice-Hall, 2002, 1119 s., Pearson International edition, 0-13-098043-9

%\bibitem{br2} MOLINA H. G. - ULLMAN J. D. - WIDOM J., 2000 , Databasse System implementation, New Jersey : Prentice-Hall, 2000, 653s., ???

%\bibitem{br3} ULLMAN J. D. - WIDOM J., 1997, A First Course in Database Systems, New Jersey : Prentice-Hall, 1997, 470s., 

%\bibitem{br4} PREFUSE, 2007, The Prefuse visualization toolkit,  [online] Dostupné na internete: <http://prefuse.org/>

%\bibitem{br5} PREFUSE Forum, Sourceforge - Prefuse Forum,  [online] Dostupné na internete: <http://sourceforge.net/projects/prefuse/>

%\end{thebibliography}

%---koniec Referencii

% -------------------
%--- Prilohy---
% -------------------

%Nepovinná časť prílohy obsahuje materiály, ktoré neboli zaradené priamo  do textu. Každá príloha sa začína na novej strane.
%Zoznam príloh je súčasťou obsahu.
%
%\addcontentsline{toc}{chapter}{Appendix A}
%\input AppendixA.tex
%
%\addcontentsline{toc}{chapter}{Appendix B}
%\input AppendixB.tex

\end{document}






