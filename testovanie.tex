\chapter{Porovnanie optimalizačných metód}
V predchádzajúcej kapitole sme popísali, čo je to problém množinového pokrytia, a
ako ho vieme využiť pre výber takej sady génov, ktorá nám zobrazí všetky udalosti, ktoré sa odohrali počas evolučnej histórie.
Náš program obsahuje greedy algoritmus, ktorý nájde približné riešenie daného problému.
Inou možnosťou je, previesť problém množinového pokrytia na celočíselný lineárny program, ktorého vyriešenia nájde optimálne riešenie.
Pre greedy algoritmus poznáme jeho aproximačný koeficient \(\ln m - \ln \ln m\ +\Theta(1) \) \cite{Slavik} 
kde \(m = |U|\). 
Všetky problémy množinového pokrytia, ktoré riešime majú spoločnú črtu, a to že vznikli ako prepis evolučnej histórie.
Toto špecifikum môže mať vplyv na to, ako blízko k optimálnemu riešeniu sa priblíži greedy algoritmus.
V tejto kapitole sa budeme venovať tomu, aké sú reálne rozdiely vo výsledkoch medzi aproximáciou a optimálnym riešením.
Popíšeme spôsob akým testy prebiehajú, pozrieme sa na vyprodukované dáta, a porovnáme výsledky pre greedy algoritmus a celočíselné lineárne programovanie.

\section{Priebeh testovania}
Priebeh testovania rozdelíme na tri časti. 
Prvou časťou je vytvorenie vstupných dát, na tento účel slúži pomocný program ktorý vytvorí požadované evolučné histórie.
Druhým krokom je spracovanie týchto histórii, to znamená vyriešenie problému množinového pokrytia pre každú z nich pomocou 
greedy algoritmu ako aj pomocou celočíselného lineárneho programovania. Záverečná časť spočíva v zozbieraní získaných dát a ich analýze.
\subsection{Generovanie evolučných histórií}
\label{sub:generovanie}

\begin{table}[!htb]
\label{tab:generator}
\begin{center}
\begin{tabular}{lcl}
node\_numb & 10 & \\
gene\_numb & 10 & \\
len\_rate & 0.75 & \\
duplication & 1 & \\ 
inversion & 1 & \\ 
deletion & 1 & 3 \\ 
insertion & 1 & 5 \\ 
transposition & 1 & \\
\end{tabular}
\end{center}
\caption{Ukážka vstupu pre generátor histórií, súbor settings.generator}
\end{table}

Pre tento účel sme zostrojili jednoduchý program umožňujúci generovanie evolučných histórií.
Program dostane na vstupe súbor s údajmi ako v tabuľke \ref{tab:generator}.
Node\_numb udáva koľko udalostí sa bude vo vygenerovanej histórii nachádzať
a gene\_numb určuje počet počiatočných génov. Každý krok evolučnej histórie bude obsahovať jednu udalosť, 
tú umiestnime na náhodné miesto do sekvencie génov,
jej dĺžku, teda to koľko génov sa v tejto udalosti nachádza vypočítame na základe hodnoty len\_rate.
V udalosti sa určite nachádza aspoň jeden gén, každý ďalší pridáme s pravdepodobnosťou len\_rate, jedná sa teda o geometrickú postupnosť.
To, aká udalosť sa v danom kroku odohrá, určíme náhodne. Prvá hodnota za každou udalosťou z tabuľky \ref{tab:generator}, 
vyjadruje pomer, s akým sa daná udalosť vyberie. V našom prípade sú udalosti v pomere 1:1:1:1:1 mali by byť teda zastúpené rovnomerne.
Pokiaľ niektorú z udalostí v histórii nechceme, nastavíme jej hodnotu 0. Pre inzerciu a deléciu vieme navyše nastaviť maximálny počet génov,
ktoré ovplyvnia, 
keďže sú to udalosti ktoré menia počet génov v kroku (Duplikácia taktiež mení počet génov, ale vytvára iba kópie, nepridáva žiadne nové gény).
Pri transpozícii gény presunieme na náhodné miesto.
Okrem vstupného súboru, môžme generátoru zadať aj počet, koľko histórií má pre dané parametre vytvoriť.
\paragraph{Spracovanie vygenerovaných histórií}
Pre všetky vygenerované histórie následne spustíme hlavný program, a vyexportujeme riešenie problému množinového pokrytia 
pomocou greedy algoritmu.
Ďalej vyexportujeme formuláciu celočíselného lineárne programu, a jeho riešenie pomocou CPLEXu.
Pre každú históriu si zapamätáme, aké množstvo blokov sme pokrývali a koľko génov sa na to mohlo vuyžiť.
Aký počet z týchto génov bol zvolený do greedy riešenia a aký do optimálneho riešenia CPLEXom.
Posledným údajom budú časy, čas potrebný pre načítanie evolučnej histórie, čas ktorý potreboval greedy algoritmus pre nájdenie riešenia a čas pri hľadaní optima.
Pre všetky histórie, ktoré generátor vytvoril z rovnakých nastavení, dáme tieto údaje dokopy. Získame priemerné hodnoty pre dané nastavenia generátora. 
To nám umožní sledovať vplyv nastavení generátora na výsledky testovania.
\subsection{Ako zopakovať testovanie ?}
\label{sub:akotestovat}

\begin{table}[t]
\label{tab:vyslednedata}
\begin{center}
%\hspace*{-1cm}
\begin{tabular}{llllllllll}
udalosti&gény&p&id&univerzum&subsety&greedy&greedy čas&ilp&ilp čas\\
\hline
10&10&0.8&1&32&12&9&0.9400&9&0.9500\\
10&10&0.8&2&42&20&15&0.9700&15&0.9400\\
10&10&0.8&3&40&20&14&0.9600&13&1.0000\\
10&10&0.8&4&30&10&6&0.9400&6&0.9600\\
10&10&0.8&5&31&11&7&0.9300&7&0.9600\\
10&10&0.8&6&38&20&16&0.9800&16&0.9500\\
10&10&0.8&7&38&16&12&0.9400&12&0.9400\\
10&10&0.8&8&34&16&14&0.9300&13&0.9700\\
10&10&0.8&9&34&10&10&0.9200&10&0.9100\\
10&10&0.8&10&37&17&15&0.9000&15&0.9500\\
10&10&0.8&avg&35.60&15.20&11.80&0.9410&11.60&0.9530\\
\end{tabular}
\end{center}
\caption{Ukážka získaných dát zo súboru file.data, pre jedny nastavenia generátora file.generator}
\end{table}

Ak má čitateľ záujem zopakovať testovanie, tu je návod ako na to.
K tejto práci, v priečinku \emph{Testovanie} prikladám všetky súbory potrebné pre zopakovanie testovania.
Konkrétne sú to tri java programy: EHDraw.jar je hlavný program popísaný v tejto práci, spracováva histórie, hľadá greedy riešenie a exportuje zadanie celočíselného lineárneho programu.
Generator.jar je popísaný v sekcii \ref{sub:generovanie}, na vstup dostáva súbor podobný tomu v tabuľke \ref{tab:generator} a údaj koľko histórií z neho má vytvoriť.
Pre zadaný vstup $filename.generator$ a počet $n$ vytvorí histórie $filename\#1.history$,$filename\#2.history$ .. $filename\#n.history$.
TestParser.jar je slúži k spracovaniu údajov pre všetky histórie ktoré vznikli z rovnakých nastavení generátora. Zoradí ich do CVS tabuľky - 
tabuľka v ktorej sú hodnoty oddelené medzeramy alebo iným znakom, v našom prípade čiarkami,
každá história bude v jednom riadku, plus sa tu nachádza riadok $avg$ s priemernými hodnotami, tak ako to môžme videť v tabuľke \ref{tab:vyslednedata}.
Okrem týchto troch programov prikladám aj program ToGenerator.jar, vygeneruje súbory *.generator, ktoré boli použité v našom testovaní.
Automatizáciu testovania zabezpečuje Linuxový príkaz \emph{make}. Vzťahy medzi súbormi, a spôsob akým \emph{make} vytvára nové súbory je popísaný v súbore \emph{Makefile}.
Na začiatku musíme existovať súbor $filename.generator$ . Prvé zavolanie príkazu \emph{make} pre každý takýto súbor spustí generátor histórií.
Údaj o tom, koľko histórií sa vygeneruje pre jedny nastavenia sa nachádza v \emph{Makefile}.
Druhé zavolanie príkazu \emph{make} pre každú históriu $filename\#i.history$ vytvorenú v predchádzajúcom kroku, nájde greedy riešenie, optimálne riešenie za pomoci CPLEXu,
a všetky výsledné údaje podstatné pre testovanie uloží do súboru $filename\#i.grepped$.
Tretie zavolanie príkazu \emph{make} pre každý súbor $filename.generator$, z ktorého sme generovali histórie $filename\#1.history$,$filename\#2.history$ .. $filename\#n.history$, vytvorí \emph{filename.data}.
V tomto súbore sa nachádza tabuľka, podobná tabuľke \ref{tab:vyslednedata}, vytvorená programom \emph{TestParser.jar}.
Prvý stĺpec je počet krokov ktoré sa odohrali, nasleduje počet génov v kroku ``root'', pravdepodobnosť s akou sme gén pridali, id súboru s históriou, veľkosť univerza,
počet podmnožín ktorými pokrývame (t.j. počet všetkých génov ktoré sa nachádzajú v histórii),počet prvkov v riešení a čas pre greedy a lineárne programovanie.
Následné spracovanie dát prebieha manuálne, na základe informácií ktoré poznáme o vstupe pre generátor.

\subsection{Výsledky testovania}
Dáta, ktoré analyzujeme, sme získali postupom uvedením v v sekcii \ref{sub:akotestovat}.
Jedná sa o celkovo 475 rôznych vstupných súborov $*.data$, v tomto súbore je každý riadok jedna evolučná história, my budeme pracovať s riadkami ``avg'', 
v ktorých sa nachádzajú priemerné hodnoty pre tieto histórie.
Pokúsim sa uviesť niekoľko výsledkov, ktoré sú podstatné pre túto prácu.

\begin{figure}[h]
 \centering
\includegraphics[width=1\textwidth]{images/cas}
\caption{Pomer času greedy a clp, pri hľadaní riešenia}\label{obr:cas}
\end{figure}


V kapitole \ref{chapter:setcover}, sa spomína, že greedy algoritmus nájde aproximáciu NP-ťažkého problému množinového pokrytia v polynomiálnom čase.
Preskúmame, či greedy algoritmus predstavuje časovú úsporu, a ak áno ako významnú.
Na obrázku \ref{obr:cas} vidíme pomer času potrebného pre riešenie greedy, vzhľadom k času potrebnému pre riešenie pomocou Celočíselného lineárneho programu.
Záznamy sú na osi $x$ zoradené vzostupne podľa toho, koľko času potreboval greedy algoritmus spolu s Celočíselným lineárnym programovaním (CLP) pre nájdenie riešenia.
Pri krátkych vstupoch nájdeme riešenie rýchlejšie s \emph{CLP}, než s greedy algoritmom, zhruba v polovici záznamov sa tento pomer začína meniť. 
Pri najkrajnejších záznamoch, ktoré vidíme aj v tabuľke \ref{tab:vyslednedata}, dostávame pre greedy a \emph{CLP} veľmi nesúrodé časy.
Pokiaľ preskúmame časy, pre jednotlivé histórie v pozadí týchto záznamov, zistíme, že väčšina času pri riešení \emph{CLP} bola využitá programom EHDraw na načítanie histórie a vytvorenie
celočíselného lineárneho programu. \emph{CPLEXu} pre takéto histórie, ktoré sú aj po optimalizácii príliš rozsiahle na zobrazenie naším programom,
stačil na nájdenie riešenia iba zlomok z celkového času.

\begin{table}[t]

\label{tab:data}
\begin{center}
%\hspace*{-1cm}
\begin{tabular}{llllllll}
poradie&udalosti&gény&p&univerzum&subsety&greedy čas&ilp\\
\hline
1&10&10&0.9&22.3&10&0.75&0.75\\
2&10&10&0.7&24.6&10&0.759&0.757\\
.\\
80&100&10&0.9&225.4&10&0.949&0.979\\
81&10&50&0.7&38.5&55.9&1.063&0.87\\
.\\
407&200&100&0.97&3060.3&1440.4&11.234&11.443\\
408&100&500&0.95&298.4&500&12.136&10.834\\
.\\
472&200&500&0.97&5609.5&5709.75&64.88&90.0225\\
473&200&500&0.97&594.67&500&74.6933&81.7333\\
474&200&500&0.97&596.9&500&86.937&82.496\\
475&200&50&0.97&593.1&50&129.09&65.178\\
\end{tabular}
\end{center}
\caption{Vybraných 10 zo 475 riadkov ``avg'', zoradené sú podla súčtu času greedy a ilp. }
\end{table}

Druhou vlastnosťou greedy algoritmu ktorá nás zaujíma je, vzdialenosť jeho riešení od optima.

Graf na obrázku \ref{obr:greedy} je zostrojený zo záznamov $all*.data$, v tých sa nachádzajú dáta z histórií,
v ktorých boli povolené všetky udalosti. Na osi $x$ sa nachádza pomer medzi medzi množstvom blokov, ktoré musíme pokryť, a počtom génov,
s ktorými tieto bloky pokrývame. Modrá čiara \emph{greedy \%} vyjadruje percentuálne, 
aký počet génov sa nachádza v greedy riešení vzhľadom na optimum, odčítavame ju z hodnôt na ľavej strane. Žltá čiara \emph{greedy avg} je priemerná hodnota tohto údaju,
$greedy\ avg=100.177437\%$, odčítavame ju z hodnôt na ľavej strane. Oranžová čiara \emph{clp \%} udáva, aké percento z génov sa nachádza v optimálnom riešení,
odčítavame ju z hodnôt na pravej strane. 
S pomocou \emph{clp \%} vidíme, že pokiaľ máme príliš veľa blokov, vzhľadom k počtu génov, využijeme pri riešení problému množinového pokrytia väčšinu génov.
V takom prípade prestáva byť optimalizácia efektívna, a aproximácia sa približuje k optimu, lebo obe riešenia potrebujú vybrať skoro všetky gény.
Naopak, pokiaľ si optimum postačí s malou časťou génov, narastá priestor pre odchýlku aproximácie od optima.
Z grafu ďalej vidíme, že aproximácia je v zaznamenanom maxime 101.74\% a priemerne 100.18\%.
Táto odchýlka aproximácie od ptima sa v praxi, t.j. pri evolučných históriách, ktoré sú zobraziteľné naším programom, prejaví rozdielom ktorý je zanedbateľný.


\begin{figure}[h]
 \centering
\includegraphics[width=1\textwidth]{images/greedy}
\caption{Graf zobrazujúci aké percento génov sa nachádza v optimálnom riešení, v závislosti od priemerného počtu unikátnych blokov na jeden gén. (pre 10 blokov a 100 génov pripadá 0.1 bloku na gén)
V grafe sa nachádza aj údaj o tom, ako ďaleko od tohto optima je aproximácia.}\label{obr:greedy}
\end{figure}
