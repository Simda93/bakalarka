\chapter{Úvod do problematiky}
V tejto kapitole si vysvetlíme základne pojmy potrebné pre túto bakalársku prácu.
\section{Biologické pozadie}
\subsection{DNA,Gén,Genóm}
\paragraph{DNA}
Deoxyribonukleová kyselina je nositeľom genetickej informácie bunky.\\
Má štruktúru dvojzávitnice, skladajúcej sa z dvoch komplementárnych vlákien.
Vlákno je tvorené nukleotidmy, ktoré obsahujú jednu zo štyroch báz Adenín,Guanín,Tymín a Cytozín.
DNA zvykneme zapisovať ako postupnosť týchto báz, kde každú bázu kódujeme jej počiatočným písmenom - A,G,T,C.

\paragraph{Gén}
Gén je súvislý úsek DNA ktorý kóduje tvorbu proteínu. Gén je základnou jednotkou dedičnosti.

\paragraph{Genóm}
Genóm je súbor DNA molekúl organizmu, ktoré sa väčšinou nachádzajú v chromozómoch.
Napríklad v ľudskom tele sa jedná o 46 molekúl DNA, jedna v každom chromozóme[6].
\subsection{Evolučná história}\label{evhist}
Evolučná história je postupnosť udalostí, ktoré sa odohrali na nejakej DNA sekvencii.
Pre potreby tejto práce sú podstatné udalosti odohrávajúce sa na dlhých úsekoch DNA - génoch.
\newline
Možné udalosti sú:
\newline
\begin{itemize}
\item \emph{Duplikácia} - skopírovanie génu na iné miesto v DNA.
\item \emph{Inzercia} - vloženie nového génu.
\item \emph{Delécia} - odstránenie génu.
\item \emph{Inverzia} - zmena poradia a orientácie génu alebo génov.
\item \emph{Translokácia} - zmena poradia génu alebo génov 
\item \emph{Speciácia} - špeciálna udalosť, ktorá označuje vznik nového druhu. Vzniká nová vetva v evolučnej histórii.
\end{itemize}
\subsubsection{Krok evolučnej histórie}
Krok evolučnej histórie, ďalej len \emph{krok e.h} je pre nás známa sekvencia génov. 
Medzi jednotlivími krokmi došlo k jednej alebo viacerím udalostiam. 

\subsection{Fylogenetický strom}
Fylogenetický strom je grafické znázornenie, ktoré zobrazuje evolučné vzťahy medzi sadou objektov. Pokiaľ si za objekty zvolíme druhy, jedná sa o takzvaný \emph{Druhový strom}.
Jednotlivé druhy sú pospájané hranami, ktoré reprezentujú evolučný vzťah.
\newline 
Druhy, ktoré sa nachádzajú na \emph{listoch} stromu sú buď existujúce druhy, z ktorých sa nevyvinuli nové druhy, alebo vyhynuté druhy bez potomkov.  
\newline
Vnútorné vrcholy predstavujú predchodcov, o ktorých sa predpokladá že sa vyskytli počas evolúcie.
\newline
Pokiaľ je v strome známy posledný spoločný predok, nazveme ho \emph{koreň}, a takýto strom označíme ako \emph{zakorenený}.
\newline
V \emph{zakorenenom} strome je zrejmá orientácia vnútorných hrán, ktorá určuje ktorý druh sa vyvinul z ktorého.
\section{Ostatné pojmy}
